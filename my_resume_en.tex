%!TEX TS-program = lualatex
\documentclass[a4paper]{comcv}

\usepackage[english]{babel}

\title{CV - Wojciech Olech}
\fullname{Wojciech}{Olech}{MSc. computer science}
\cvtitle{High- and low-level developer, also likes electronics}
\website{https://steelph0enix.github.io}{steelph0enix.github.io}
\email{wojciech\_olech@hotmail.com}
\github{https://github.com/SteelPh0enix}{GitHub: SteelPh0enix}
\linkedin{https://www.linkedin.com/in/wojciech-olech-9b6654140}{LinkedIn}
\currentdate{07.03.2024}

\begin{document}

C/C++/Python developer for around a decade. I've been also working with electronics and microcontrollers (AVR and then ARM) for around 8 years. My career started with university projects and commissions for telemedicine-related prototypes, where i've been (both in a team, and solo) responsible for developing low-level firmware and testing hardware for embedded solutions. Last year i've been evaluating Rust language for it's viability in space applications and maintaining critical C software for space missions (Board Support Package and Boot Software).

\vspace{\topsep}

\section{Experience}

\combosection{N7 Space}{Embedded developer}{10.2021-now}{
    \begin{tightlist}
        \item Maintaining and refactoring modules for critical space-grade software
        \item Creating Hardware Abstraction Layer for SAMV71 MCU in Rust
        \item Updating the test environment for ECSS-compliant CANopen library stack. 
    \end{tightlist}
}

\vspace{\topsep}

\combosection{ActiveLife}{Main embedded developer}{09.2021-05.2022}{
    \begin{tightlist}
        \item Creating software for data acquisition on STM32 and it's real-time transmission to PC
        \item Implementation of \href{https://github.com/SteelPh0enix/S2LP_Driver}{S2-LP} and MAX30001 libraries for STM32 MCU
        \item Diagnosing the hardware issues of the device's prototype.
    \end{tightlist}
}

\vspace{\topsep}

\combosection{STMicroelectronics}{Technical support engineer}{03.2020-03.2022}{
    \begin{tightlist}
        \item Software and hardware-related customer support
        \item Hands-on trainings for VL53-series distance sensors.
    \end{tightlist}
}

\vspace{\topsep}

\combosection{Teleinfomed}{Main embedded developer}{06.2019-11.2019}{
    \begin{tightlist}
        \item Creating software for EEG data acquisition using ADS1299 module on STM32 microcontroller
        \item Integration of the device with phones via BLE (GATT)
        \item Integration of the device with server via REST API (WiFi)
        \item TFTP protocol implementation for STM32 and SPWF01 module
    \end{tightlist}
}

\vspace{\topsep}

\combosection{Orion Project}{Main embedded developer/Team Leader}{2016-2023}{
    Orion Project is a student project of mars rover for European Rover Challenge, created by students on Lublin University of Technology
    \vspace{\topsep}
    \begin{tightlist}
        \item Implementation of \href{https://github.com/SteelPh0enix/OrionChassisDriverV3}{chassis} and \href{https://github.com/SteelPh0enix/Orion-v3-Arm-Controller}{arm} firmware on Arduino and STM32
        \item Implementation of \href{https://github.com/SteelPh0enix/OrionGRPC}{network communications software} between the rover and base station (PC) via GRPC
        \item And many more little side-projects and utilities that would take too long to list here...
    \end{tightlist}
}

\vspace{\topsep}

\section{Private projects}

\vspace{\topsep}
\vspace{\topsep}
\vspace{\topsep}

\begin{tightlist}
    \item \href{https://github.com/SteelPh0enix/STM32_HM10_Driver}{HM-10 driver for STM32} (C++), along with custom RFComm protocol support and \href{https://github.com/SteelPh0enix/BLERFCommClient}{desktop app} (C++/Qt)
    \item \href{https://github.com/SteelPh0enix/BLEWeatherStation}{Mini weather station} based on X-NUCLEO-IKS01A3 and BlueNRG-M2SP for BLE support, paired with \href{https://github.com/SteelPh0enix/WeatherStationAndroidApp}{an android app} (Kotlin)
    \item \href{https://github.com/SteelPh0enix/Orion-Mini-Rover}{Firmware for Orion Mini rover} and \href{https://github.com/SteelPh0enix/Orion-Mini-Controller}{it's controller}, communicating via nRF24L01+ using JSON API (C++)
\end{tightlist}

\vspace{\topsep}

\section{Education}

\combosection{Electronic School Group in Lublin}{ITC Technician}{2013-2017}{}
\vspace{\topsep}
\combosection{Lublin University of Technology}{Computer Science Engineer}{2017-2021}{}
Title of thesis: Analog data acquisitor based on RaspberryPi
\vspace{\topsep}
\combosection{Lublin University of Technology}{MSc. in Computer Science}{2021-2023}{}
Title of thesis: Comparative analysis of methods for tracking mobile robotic platforms

\vspace{\topsep}

\end{document}
